\documentclass[11pt,a4paper]{article}
\usepackage[utf8]{inputenc}
\usepackage[margin=1in]{geometry}
\usepackage{listings}
\usepackage{xcolor}
\usepackage{hyperref}
\usepackage{graphicx}
\usepackage{fancyhdr}
\usepackage{amsmath}
\usepackage{amssymb}
\usepackage{booktabs}
\usepackage{enumitem}

% Code listing settings
\lstset{
    basicstyle=\ttfamily\small,
    breaklines=true,
    breakatwhitespace=true,
    columns=fullflexible,
    showstringspaces=false,
    commentstyle=\color{gray},
    keywordstyle=\color{blue},
    stringstyle=\color{orange},
    frame=single,
    numbers=left,
    numberstyle=\tiny\color{gray},
    backgroundcolor=\color{gray!5}
}

% Language definitions
\lstdefinelanguage{JavaScript}{
    keywords={const, let, var, function, return, if, else, for, while, typeof, Array, throw, new},
    sensitive=true,
    comment=[l]{//},
    morecomment=[s]{/*}{*/},
    morestring=[b]',
    morestring=[b]"
}

\lstdefinestyle{python}{
    language=Python,
    keywords={import, def, if, else, for, while, return, class, True, False, None}
}

\lstdefinestyle{cpp}{
    language=C++,
    keywords={int, char, bool, void, const, static, return, if, else, for, while, new, delete, struct, class, std, cout, endl, include}
}

% Header and footer
\pagestyle{fancy}
\fancyhf{}
\rhead{NYCU - Software Testing}
\lhead{Faulty Programs Analysis}
\rfoot{Page \thepage}

% Remove numbering for subsubsections (third level)
\setcounter{secnumdepth}{2}

\title{\textbf{Software Testing Assignment \# 1}\\ Faulty Programs Analysis}
\author{Ping-Chen Chung}
\date{\today}

\begin{document}

\maketitle
\tableofcontents
\newpage

\section{Part 1: JavaScript Programs - Fault Analysis}

This section analyzes four faulty JavaScript programs, identifying faults, proposing fixes, and exploring test cases that execute faults, reach error states, and produce failures.

\subsection{Program 1: \texttt{findLast(x, y)}}

\subsubsection{Problem Description}
The function should find the last index of element \texttt{y} in array \texttt{x}.

\subsubsection{(a) Fault Explanation and Fix}

 The fault is in the \texttt{for} loop's continuation condition. The condition \texttt{i > 0} causes the loop to stop when \texttt{i} is 0, so the element at index 0 of the array is never checked. The function will incorrectly return -1 if the element being searched for is only present at index 0.

\textbf{Modification:}
\begin{lstlisting}[language=JavaScript]
// Original (FAULTY):
for (let i = x.length - 1; i > 0; i--)

// Corrected:
for (let i = x.length - 1; i >= 0; i--)
\end{lstlisting}

\subsubsection{(b) Test Case That Does Not Execute the Fault}

This is not strictly possible, as any call with a non-empty array will execute the line of code containing the fault (the \texttt{for} loop statement). However, it is possible to provide a test case where the fault is executed but does not cause the program to fail.

\textbf{Example:} \texttt{x = [1, 2, 3]}, \texttt{y = 2}. The program correctly returns \texttt{1}.

\subsubsection{(c) Test Case That Executes Fault But No Error State}

\begin{itemize}[leftmargin=*]
    \item \textbf{Test Case:} \texttt{x = [10, 20, 30]}, \texttt{y = 20}
    \item \textbf{Explanation:} The faulty code in the \texttt{for} loop is executed. However, the value \texttt{20} is found at index \texttt{1}. The function correctly returns \texttt{1} and terminates before the loop counter \texttt{i} reaches \texttt{0}. Because the search is successful before the boundary condition is tested, the fault does not lead to an incorrect internal state.
    \item \textbf{Expected Output:} \texttt{1}
    \item \textbf{Actual Output:} \texttt{1}
\end{itemize}

\subsubsection{(d) Test Case With Error State But No Failure}

\textbf{Not Possible.}

 For this function, an error state occurs when the loop completes without finding \texttt{y} because \texttt{y} is at index 0. At this moment, the program's internal state is incorrect because it has failed to find the element. This error state immediately leads to the function returning \texttt{-1}. Since the incorrect return value \textit{is} the external incorrect behavior, the error state directly causes a failure. There are no subsequent operations that could mask or correct this error.

\subsubsection{(e) First Error State}

Not applicable, as a test case for (d) is not possible.

\subsection{Program 2: \texttt{lastZero(x)}}

\subsubsection{Problem Description}
The function should find the last index of zero in array \texttt{x}.

\subsubsection{(a) Fault Explanation and Fix}

 The function is named \texttt{lastZero}, but its implementation finds the \textit{first} index of zero. The fault is that the \texttt{for} loop iterates forward from the beginning of the array (\texttt{i = 0; i < x.length; i++}) and returns immediately upon finding the first match.

\textbf{Modification:}
\begin{lstlisting}[language=JavaScript]
// Original (FAULTY):
for (let i = 0; i < x.length; i++)

// Corrected:
for (let i = x.length - 1; i >= 0; i--)
\end{lstlisting}

\subsubsection{(b) Test Case That Does Not Execute the Fault}

This is not strictly possible, as any call with a non-empty array will execute the faulty \texttt{for} loop. However, we can provide a case where the fault does not lead to a failure, such as when the first zero is also the last zero: \texttt{x = [5, 1, 0, 8]}.

\subsubsection{(c) Test Case That Executes Fault But No Error State}

\begin{itemize}[leftmargin=*]
    \item \textbf{Test Case:} \texttt{x = [1, 5, 0, 8]}
    \item \textbf{Explanation:} The faulty forward-iterating loop is executed. However, since there is only one \texttt{0} in the array, the index of the first \texttt{0} is the same as the index of the last \texttt{0}. The program correctly returns \texttt{2}. The internal state is never incorrect, so no error state is reached.
    \item \textbf{Expected Output:} \texttt{2}
    \item \textbf{Actual Output:} \texttt{2}
\end{itemize}

\subsubsection{(d) Test Case With Error State But No Failure}

\textbf{Not Possible.}

 An error state occurs when the function finds a \texttt{0} at an early index \texttt{i} and is about to return \texttt{i}, while the true last \texttt{0} exists at a later index \texttt{j}. This incorrect internal decision (to return \texttt{i}) is inseparable from the external failure (returning the wrong index). The error cannot be corrected before the function terminates.

\subsubsection{(e) First Error State}

Not applicable, as a test case for (d) is not possible.

\subsection{Program 3: \texttt{countPositive(x)}}

\subsubsection{Problem Description}
The function should count strictly positive (greater than 0) elements in array \texttt{x}.

\subsubsection{(a) Fault Explanation and Fix}

 The function is intended to count ``positive'' elements, which are strictly greater than 0. The fault is that the condition \texttt{x[i] >= 0} incorrectly includes \texttt{0} in the count.

\textbf{Modification:}
\begin{lstlisting}[language=JavaScript]
// Original (FAULTY):
if (x[i] >= 0)

// Corrected:
if (x[i] > 0)
\end{lstlisting}

\subsubsection{(b) Test Case That Does Not Execute the Fault}

This is not strictly possible, as the faulty \texttt{if} condition is executed for each array element. However, we can provide a case that does not expose the fault by using an array that does not contain \texttt{0}: \texttt{x = [-2, 5, 10]}.

\subsubsection{(c) Test Case That Executes Fault But No Error State}

\begin{itemize}[leftmargin=*]
    \item \textbf{Test Case:} \texttt{x = [-5, 1, 9]}
    \item \textbf{Explanation:} The faulty condition \texttt{x[i] >= 0} is executed for each element. Because the input array does not contain \texttt{0}, the behavior of \texttt{>= 0} is identical to \texttt{> 0} for every element. The internal state variable \texttt{count} is always correct throughout the execution, so no error state is entered.
    \item \textbf{Expected Output:} \texttt{2}
    \item \textbf{Actual Output:} \texttt{2}
\end{itemize}

\subsubsection{(d) Test Case With Error State But No Failure}

\textbf{Not Possible.}

 An error state occurs when the program processes a \texttt{0} and incorrectly increments \texttt{count}. At this point, the \texttt{count} variable is one greater than its correct value. Since the \texttt{count} variable is only ever incremented, this error cannot be corrected. The incorrect value will persist and be returned at the end, directly causing a failure.

\subsubsection{(e) First Error State}

Not applicable, as a test case for (d) is not possible.

\subsection{Program 4: \texttt{oddOrPos(x)}}

\subsubsection{Problem Description}
The function should count elements that are either odd or positive.

\subsubsection{(a) Fault Explanation and Fix}

 The fault is in the logic used to identify odd numbers. The condition \texttt{x[i] \% 2 === 1} fails for negative odd numbers. In JavaScript, the result of the modulo operator (\texttt{\%}) on a negative number is negative (e.g., \texttt{-3 \% 2} is \texttt{-1}), so \texttt{-1 === 1} evaluates to false.

\textbf{Modification:}
\begin{lstlisting}[language=JavaScript]
// Original (FAULTY):
if (x[i] % 2 === 1 || x[i] > 0)

// Corrected:
if (x[i] % 2 !== 0 || x[i] > 0)
\end{lstlisting}

\subsubsection{(b) Test Case That Does Not Execute the Fault}

This is not strictly possible, as the faulty \texttt{if} condition is always executed. However, we can provide a test case that does not expose the fault by avoiding negative odd numbers: \texttt{x = [-2, 1, 4]}.

\subsubsection{(c) Test Case That Executes Fault But No Error State}

\begin{itemize}[leftmargin=*]
    \item \textbf{Test Case:} \texttt{x = [-2, 3, 4]}
    \item \textbf{Explanation:} The faulty condition \texttt{x[i] \% 2 === 1} is executed. Since the input contains no negative odd numbers, the faulty logic behaves correctly for all elements. \texttt{3 \% 2} is \texttt{1}, so it is counted. \texttt{-2} is not counted. \texttt{4} is counted because it's positive. The \texttt{count} variable remains correct throughout the execution, so no error state is reached.
    \item \textbf{Expected Output:} \texttt{2}
    \item \textbf{Actual Output:} \texttt{2}
\end{itemize}

\subsubsection{(d) Test Case With Error State But No Failure}

\textbf{Not Possible.}

 An error state occurs when the program processes a negative odd number (like \texttt{-3}) and fails to increment \texttt{count}. At this point, the \texttt{count} variable is less than its correct value. As \texttt{count} is never decremented, it is impossible for it to ``catch up'' or be corrected later in the execution. This persistent error in the internal state will inevitably lead to an incorrect final return value, which is a failure.

\subsubsection{(e) First Error State}

Not applicable, as a test case for (d) is not possible.

\newpage
\section{Part 2: Multi-Language Programs - Verification Framework}

This section documents the comprehensive analysis and automated verification framework built for five faulty programs demonstrating real-world software defects: deadlock, memory leaks, race conditions, buffer overflows, and const violations.

\subsection{Overview}

This document details all issues discovered in the software testing homework assignment, including incomplete source files, verifier script problems, and the solutions implemented to create a comprehensive automated testing framework.

\subsection{Issues Found in Verifier Script (\texttt{verifier.cpp})}

\subsubsection{Issue 1: Incorrect Compiler Flag}

\textbf{Location:} Line 17

\textbf{Problem:} Used \texttt{-Wno-fpermissive} which disables permissive mode warnings, causing compilation errors for intentionally faulty code.

\begin{lstlisting}[style=cpp]
// BEFORE (WRONG):
std::string command = "g++-15 -g -pthread " + source + " -o " + output
    + " -isysroot $(xcrun --show-sdk-path) -Wno-fpermissive ";

// AFTER (FIXED):
std::string command = "g++-15 -std=c++11 -g -pthread " + source + " -o " + output
    + " -isysroot $(xcrun --show-sdk-path) -fpermissive ";
\end{lstlisting}

\textbf{Impact:} ProfileUpdater.cpp and MatrixProcessor.cpp failed to compile.

\textbf{Solution:} Changed to \texttt{-fpermissive} to allow intentionally faulty conversions to compile.

\subsubsection{Issue 2: Memory Leak Detection - Wrong Output Redirection}

\textbf{Location:} Line 59

\textbf{Problem:} Redirected only stderr (\texttt{2>}) but \texttt{leaks} tool outputs to stdout.

\begin{lstlisting}[style=cpp]
// BEFORE (WRONG):
std::system("leaks -atExit -- ./data_processer 2> leaks_output.txt");

// AFTER (FIXED):
std::system("leaks -atExit -- ./data_processer > evidence/leaks_output.txt 2>&1");
\end{lstlisting}

\textbf{Impact:} Memory leak output was not captured, test always failed.

\textbf{Solution:} Redirect both stdout and stderr to capture full \texttt{leaks} output.

\subsubsection{Issue 3: Memory Leak Detection Pattern Too Strict}

\textbf{Location:} Lines 63-64

\textbf{Problem:} Only looked for ``leaked bytes'' pattern, missing ``leaks for'' pattern.

\begin{lstlisting}[style=cpp]
// BEFORE (WRONG):
if (output.find("leaked bytes") != std::string::npos &&
    output.find("0 leaks for 0 bytes") == std::string::npos)

// AFTER (FIXED):
if ((output.find("leaked bytes") != std::string::npos ||
     output.find("leaks for") != std::string::npos) &&
    output.find("0 leaks for 0") == std::string::npos)
\end{lstlisting}

\textbf{Impact:} Failed to detect leaks with format ``20000 leaks for 480000 total leaked bytes.''

\textbf{Solution:} Accept either pattern and generalize the ``no leak'' check.

\subsubsection{Issue 4: No Evidence File Organization}

\textbf{Problem:} Test output files scattered in root directory, no preservation after cleanup.

\textbf{Solution Implemented:}
\begin{enumerate}
    \item Created \texttt{evidence/} directory to store all test outputs
    \item Updated all test functions to write to \texttt{evidence/[test\_name]\_output.txt}
    \item Modified cleanup to preserve evidence files
    \item Added evidence file notification in PASS messages
\end{enumerate}

Evidence files created:
\begin{itemize}
    \item \texttt{evidence/deadlock\_output.txt}
    \item \texttt{evidence/leaks\_output.txt}
    \item \texttt{evidence/race\_output.txt}
    \item \texttt{evidence/overflow\_output.txt}
    \item \texttt{evidence/const\_violation\_output.txt}
\end{itemize}

\subsubsection{Issue 5: Testing Incomplete Source Files}

\textbf{Problem:} Verifier attempted to compile and test incomplete source files (missing main functions, syntax errors).

\textbf{Solution:} Created separate test implementation files in \texttt{tests/} directory:
\begin{itemize}
    \item \texttt{tests/test\_deadlock.py} - Complete deadlock test with main function
    \item \texttt{tests/test\_buffer\_overflow.cpp} - Complete buffer overflow test
    \item \texttt{tests/test\_const\_violation.cpp} - Complete const violation test with global const data
\end{itemize}

This approach:
\begin{itemize}
    \item Keeps original buggy source files unchanged
    \item Provides complete test implementations
    \item Safely handles crashes without affecting verifier
    \item Makes tests reproducible
\end{itemize}

\subsection{Issues Found in Source Files}

\subsubsection{Program 1: ResourceScheduler.py - Deadlock}

\paragraph{Issues Found:}

\begin{enumerate}
    \item \textbf{Line 19:} Syntax error - incomplete assignment
\begin{lstlisting}[style=python]
task_queue =  # SYNTAX ERROR
\end{lstlisting}

    \item \textbf{Lines 79-81:} Empty main block
\begin{lstlisting}[style=python]
if __name__ == "__main__":
    # Empty - no code to start threads
\end{lstlisting}
\end{enumerate}

\paragraph{Solution:}
Created \texttt{tests/test\_deadlock.py} with complete implementation including:
\begin{itemize}
    \item Fixed \texttt{task\_queue = []}
    \item Complete main block that starts both threads
    \item Timeout mechanism to detect deadlock
\end{itemize}

\paragraph{Verification Evidence:}
\begin{verbatim}
# From evidence/deadlock_output.txt
2025-10-02 01:16:23,456 | Thread-A | Thread-A is starting.
2025-10-02 01:16:23,456 | Thread-A | Thread-A attempting to acquire lock on Resource A...
2025-10-02 01:16:23,456 | Thread-A | Thread-A acquired lock on Resource A.
                                       Waiting for Resource B...
2025-10-02 01:16:23,457 | Thread-B | Thread-B is starting.
2025-10-02 01:16:23,457 | Thread-B | Thread-B attempting to acquire lock on Resource B...
2025-10-02 01:16:23,457 | Thread-B | Thread-B acquired lock on Resource B.
                                       Waiting for Resource A...
# Program hangs here - DEADLOCK detected by timeout
\end{verbatim}

\subsubsection{Program 2: ProfileUpdater.cpp - Buffer Overflow}

\paragraph{Issues Found:}

\begin{enumerate}
    \item \textbf{Line 11:} Misleading comment
\begin{lstlisting}[style=cpp]
char username[1]; // Buffer of 20 characters  <- WRONG! Only 1 byte
\end{lstlisting}

    \item \textbf{Line 20:} Wrong memset usage
\begin{lstlisting}[style=cpp]
memset(profile_status, 0, sizeof(profile_status));  // Invalid conversion char to void*
\end{lstlisting}
Should be: \texttt{memset(\&profile\_status, 0, sizeof(profile\_status));} or just \texttt{profile\_status = '\textbackslash 0';}

    \item \textbf{Lines 48-50:} Empty main function
\begin{lstlisting}[style=cpp]
int main() {
    // trigger the fault
    return 0;
}
\end{lstlisting}
\end{enumerate}

\paragraph{Solution:}
Created \texttt{tests/test\_buffer\_overflow.cpp} with:
\begin{itemize}
    \item Corrected memset call
    \item Complete test that creates profile, sets fields, triggers overflow, shows corruption
    \item Long string that overflows 1-byte buffer
\end{itemize}

\paragraph{Verification Evidence:}
\begin{verbatim}
# From evidence/overflow_output.txt
--- Profile Before Update ---
Username:
User ID: 12345
Status: A
Is Active: Yes
Last Login: 2024
----------------------------

Attempting to update with a malicious, oversized username...

--- Profile After Malicious Update ---
Username: ThisIsAVeryLongUsernameThatWillCauseABufferOverflow
User ID: 1936269938  # CORRUPTED! Was 12345
Status: <garbage>     # CORRUPTED! Was 'A'
Is Active: Yes
Last Login: 1702129518  # CORRUPTED! Was 2024
\end{verbatim}

\textbf{Result:} Memory corruption confirmed - adjacent struct members overwritten.

\subsubsection{Program 3: MatrixProcessor.cpp - Const Violation}

\paragraph{Issues Found:}

\begin{enumerate}
    \item \textbf{Line 27:} Invalid syntax
\begin{lstlisting}[style=cpp]
non_const_matrix = 999;  // Invalid conversion from 'int' to 'int**'
\end{lstlisting}
Should be: \texttt{non\_const\_matrix[0][0] = 999;}

    \item \textbf{Lines 30-34:} Empty main function
\begin{lstlisting}[style=cpp]
int main() {
    // trigger the fault
    return 0;
}
\end{lstlisting}
\end{enumerate}

\paragraph{Solution:}
Created \texttt{tests/test\_const\_violation.cpp} with:
\begin{itemize}
    \item \textbf{Critical fix:} Used global \texttt{static const} arrays instead of local const arrays
    \item Local const arrays are on the stack (writable)
    \item Global const arrays are in read-only memory segment
    \item Attempting to modify triggers segmentation fault
\end{itemize}

\begin{lstlisting}[style=cpp]
// Global const data - placed in read-only memory
static const int row0[] = {10, 20, 30};
static const int row1[] = {40, 50, 60};

int main() {
    const int* const matrix[] = {row0, row1};
    process_matrix(matrix, 2, 3);  // CRASH!
}
\end{lstlisting}

\paragraph{Verification Evidence:}
\begin{verbatim}
# From evidence/const_violation_output.txt
Value before calling function: matrix[0][0] = 10

Calling function that will attempt to modify const data...
Inside process_matrix function.
Processing a 2x3 matrix.
Attempting to modify const data (undefined behavior)...
# SEGMENTATION FAULT - Program crashed (exit code 138)
\end{verbatim}

\textbf{Result:} Program crashed as expected when attempting to write to read-only memory.

\subsubsection{Program 4: LoggingSystem.cpp - Race Condition}

\paragraph{Status:} \checkmark Complete - No issues found

This file was complete and functional. The race condition is inherent in the design:

\begin{lstlisting}[style=cpp]
void processLogs(int thread_id, int num_logs_to_process) {
    for (int i = 0; i < num_logs_to_process; ++i) {
        simulateLogProcessing();
        total_logs_processed++;  // NO MUTEX - Race condition here
    }
}
\end{lstlisting}

\paragraph{Verification Evidence:}
\begin{verbatim}
# From evidence/race_output.txt
Starting logging system test...
Processing logs across 10 threads...
All threads completed.

Expected total logs: 100000
Final count of logs processed: 98320  # Lost 1680 updates due to race!
\end{verbatim}

\textbf{Result:} Race condition confirmed - final count consistently less than expected.

\subsubsection{Program 5: DataProcessor.cpp - Memory Leak}

\paragraph{Issues Found:}

\textbf{Line 39:} Wrong array allocation
\begin{lstlisting}[style=cpp]
newRecord->data_buffer = new char;  // Only allocates 1 byte!
\end{lstlisting}
Should be: \texttt{newRecord->data\_buffer = new char[bufferSize];}

However, this doesn't affect leak detection since the memory is still leaked either way.

\paragraph{Status:} \checkmark Functionally Complete

The main function properly triggers the leak by calling \texttt{startDataIngestion()} which allocates 10,000 records without cleanup.

\paragraph{Verification Evidence:}
\begin{verbatim}
# From evidence/leaks_output.txt
Process 65191: 20190 nodes malloced for 580 KB
Process 65191: 20000 leaks for 480000 total leaked bytes.

STACK OF 10000 INSTANCES OF 'ROOT LEAK: <malloc in processLargeFile>':
5   dyld                                  0x195539d54 start + 7184
4   data_processer                        0x1021c0b90 main + 12
3   data_processer                        0x1021c0b1c startDataIngestion() + 208
2   data_processer                        0x1021c0944 processLargeFile(...) + 184
1   libstdc++.6.dylib                     0x1026c85fc operator new(unsigned long) + 28
0   libsystem_malloc.dylib                0x195725f00 _malloc_zone_malloc + 152

20000 (469K) << TOTAL >>
\end{verbatim}

\textbf{Result:} Massive memory leak confirmed:
\begin{itemize}
    \item 10,000 \texttt{DataRecord} objects never freed
    \item 10,000 \texttt{data\_buffer} arrays never freed
    \item Total: 20,000 leaks for 480KB
\end{itemize}

\subsection{Project Organization}

\subsubsection{Directory Structure}

\begin{verbatim}
hw1/
├── prob_src/               # Original faulty source files
│   ├── ResourceScheduler.py
│   ├── DataProcessor.cpp
│   ├── LoggingSystem.cpp
│   ├── ProfileUpdater.cpp
│   └── MatrixProcessor.cpp
├── tests/                  # Test implementations
│   ├── test_deadlock.py
│   ├── test_buffer_overflow.cpp
│   └── test_const_violation.cpp
├── evidence/               # Test output evidence (preserved)
│   ├── deadlock_output.txt
│   ├── leaks_output.txt
│   ├── race_output.txt
│   ├── overflow_output.txt
│   └── const_violation_output.txt
├── verifier.cpp            # Main verification framework
├── prob.js                 # JavaScript homework (findLast, etc.)
├── prob_test.js            # Node.js tests for JavaScript homework
└── data.txt                # Input data for DataProcessor
\end{verbatim}

\subsection{Verification Results Summary}

\begin{table}[h]
\centering
\begin{tabular}{@{}lllp{5cm}@{}}
\toprule
\textbf{Test} & \textbf{Status} & \textbf{Evidence} & \textbf{Key Finding} \\ \midrule
Deadlock &  PASS & Program timeout (5s) & Both threads blocked waiting for each other's locks \\
Memory Leak &  PASS & 20000 leaks detected & 480KB leaked across 10,000 records \\
Race Condition &  PASS & 98320/100000 processed & Lost 1680 updates due to concurrent access \\
Buffer Overflow &  PASS & Memory corruption & Adjacent struct fields overwritten \\
Const Violation &  PASS & SEGFAULT (exit 138) & Crash writing to read-only memory \\ \bottomrule
\end{tabular}
\caption{Verification Test Results}
\end{table}

\subsection{Technical Insights}

\subsubsection{Buffer Overflow Details}

The overflow works because of struct memory layout:

\begin{verbatim}
Memory Layout of UserProfile:
[username: 1 byte][padding: 3 bytes][user_id: 4 bytes][profile_status: 1 byte]
[padding: 3 bytes][is_active: 1 byte][padding: 3 bytes][last_login_year: 4 bytes]

When strcpy writes 52 bytes into 1-byte buffer:
- Overwrites padding
- Overwrites user_id
- Overwrites profile_status
- Overwrites is_active
- Overwrites last_login_year
- May crash or corrupt further memory
\end{verbatim}

\subsubsection{Const Violation Key Learning}

\textbf{Why local const didn't crash but global const did:}
\begin{itemize}
    \item \textbf{Stack variables:} Even if declared const, compiler may place on writable stack
    \item \textbf{Global/static const:} Placed in \texttt{.rodata} section (read-only data segment)
    \item \textbf{Attempting to write to .rodata:} Hardware memory protection triggers SEGFAULT
\end{itemize}

\subsubsection{Race Condition Math}

With 10 threads doing 10,000 increments each:
\begin{itemize}
    \item Expected: 100,000
    \item Typical result: 98,000-99,000
    \item Lost updates: 1,000-2,000 (1-2\% race window)
\end{itemize}

The race window exists during the read-modify-write cycle:

\begin{verbatim}
LOAD  total_logs_processed -> register  # Thread A reads
INC   register                          # Thread A increments
LOAD  total_logs_processed -> register  # Thread B reads (same value!)
STORE register -> total_logs_processed  # Thread A writes
STORE register -> total_logs_processed  # Thread B writes (overwrites A!)
\end{verbatim}

\subsection{Compilation and Execution}

\subsubsection{Compile Verifier}
\begin{lstlisting}[language=bash]
g++-15 -std=c++11 verifier.cpp -o verifier
\end{lstlisting}

\subsubsection{Run All Tests}
\begin{lstlisting}[language=bash]
./verifier
\end{lstlisting}

\subsubsection{Individual Test Compilation}
\begin{lstlisting}[language=bash]
# Deadlock test
python3 tests/test_deadlock.py

# Buffer overflow test
g++-15 -std=c++11 -fpermissive tests/test_buffer_overflow.cpp -o test_buffer_overflow
./test_buffer_overflow

# Const violation test
g++-15 -std=c++11 -fpermissive tests/test_const_violation.cpp -o test_const_violation
./test_const_violation

# Race condition test
g++-15 -std=c++11 -pthread prob_src/LoggingSystem.cpp -o logging_system
./logging_system

# Memory leak test
g++-15 -std=c++11 prob_src/DataProcessor.cpp -o data_processer
leaks -atExit -- ./data_processer
\end{lstlisting}

\section{Conclusion}

All five faulty programs have been successfully verified with comprehensive test cases. The automated verification framework now:


\textbf{Final Verification Summary:}
\begin{verbatim}
Passed:  5
Failed:  0
Skipped: 0 (with gtimeout installed)
\end{verbatim}

All evidence files preserved in \texttt{evidence/} directory for review and grading.

\end{document}
